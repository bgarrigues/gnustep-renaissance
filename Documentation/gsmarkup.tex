\chapter{The GNUstep Markup Language}
\section{The GNUstep Markup Language}

The GNUstep Renaissance Markup Language (gsmarkup for short) is used to
describe graphical user interfaces (windows, menus, etc).  A gsmarkup
file contains a description of how the graphical user interface should
be created, which objects are to be placed in the user interface, and
how these objects are interconnected between them and to other objects
in the main application.

\section{General overview of the main components}

\subsection{XML Header}
A well formed gsmarkup file begins with the standard XML declaration,
\begin{verbatim}
<?xml version="1.0"?>
<!DOCTYPE gsmarkup>
\end{verbatim}
While this declaration is not strictly necessary, it's pretty to have
-- it declares that the file is an XML version 1 file, and that the
document type is \texttt{gsmarkup}.

\subsection{The \texttt{<gsmarkup>} tag}
The document begins with a \texttt{<gsmarkup>} tag, and ends with a 
\texttt{</gsmarkup>}.  So, here is a skeleton gsmarkup file:
\begin{verbatim}
<?xml version="1.0"?>
<!DOCTYPE gsmarkup>
<gsmarkup>
  <!-- the document goes here -->
</gsmarkup>
\end{verbatim}
(this \texttt{<gsmarkup>} tag is the equivalent of the \texttt{<html>}
tag in HTML.

\subsection{Comments}
Comments can be inserted anywhere in the file inside a \texttt{<!--
-->} tag.  For example,
\begin{verbatim}
<!-- This is a comment -->
\end{verbatim}
Such comments are ignored by the parser.

\subsection{Special Characters}
A few characters are special XML characters.  To insert these
characters inside strings or free text in the file, you need to use
the appropriate XML entities.  The table \ref{XML-entities} lists
these important entities for easy reference.
\begin{table}
\caption{\label{XML-entities}Special XML entities}
\begin{tabular}{l|c}
Entity  & Value\\
\hline
\&amp;  & \texttt{\&} \\
\&apos; & \texttt{'} \\
\&gt;   & \texttt{>} \\
\&lt;   & \texttt{<} \\
\&quot; & \texttt{"} \\
\end{tabular}
\end{table}
For example, to insert \texttt{<} in the title of a button, you need
to write it as \texttt{\&lt;}.

\subsection{\texttt{<objects>} and \texttt{<connectors>} sections}

Inside the \texttt{<gsmarkup>} document tag, you can place objects and
connectors section.  An objects section is started by opening a
\texttt{<objects>} tag, and ended by closing it.  A connectors section
is started by opening a \texttt{<connectors>} tag, and ended by
closing it.  You can use as many objects and connectors sections as
you want, even if normally just one objects and one connectors
sections are present.  You can not nest objects and connectors
sections.  An objects section declares a list of objects which are to
be created when the file is loaded into a running application.
Typically, the list of objects might include windows (and all the
widgets contained in the windows), menus, and other arbitrary objects,
including objects of custom classes you have defined in your
application.  The connectors section declares a list of connectors;
each connector represents a connection to establish between two
objects - typically between an object residing in your application and
one of the objects created when the file is loaded.  As we will
explain later on, connections between objects created when the file is
loaded (and connections from an object within the file and an external
object in your application) can be declared using a special syntax
inside the objects section itself, so normally the connectors section
is very short, and it is mostly provided for the special cases which
can't be managed by the other syntax.  Programs generating gsmarkup files
are expected to put all connections in the objects section, except
connections which can't be embedded there.

The difference between the two section types is important because the
parser behaves differently inside an objects section and inside a
connectors section.

To let you get a more practical feeling of the thing, here is a very
simple gsmarkup file creating a button:
\begin{verbatim}
<?xml version="1.0"?>
<!DOCTYPE gsmarkup>
<gsmarkup>

<objects>

  <window title="Button Test" closable="NO" frameAutosaveName="main" >
    <button title="Click this button to quit" action="terminate:"/>
  </window>

</objects>

</gsmarkup>
\end{verbatim}
We will explain the \texttt{window} and the \texttt{button} tags later
on.  Just notice that in this example, there is only an objects
section in the file (which creates a window containing a button), and
there is no connectors section.

If you want to run this file, save it as a \texttt{button.gsmarkup} file,
then type at the command prompt
\begin{verbatim}
openapp GSMarkupBrower.app button.gsmarkup
\end{verbatim}
(of course you need to have installed Renaissance for this to work).  You
can normally load any file to see how it looks by using
\texttt{GSMarkupBrowser.app} in this way.

\section{The objects section}

\subsection{Relationship between tags and objects}

The objects section describes objects which are to be created when the
file is loaded into a running application.  The section describes the
intention of the programmer about the objects to be created -- the
logic of the GUI objects in other words -- rather than the precise
details of how the objects are to be created on each platform.  For
example, a button with title 'Quit' and which, when clicked,
terminates the application, is described in the file as:
\begin{verbatim}
<button title="Quit" action="terminate:"/>
\end{verbatim}
The actual appearance of the button (border, imaged used for borders,
font used for the title, changing in the button appearance when it's
clicked etc) is not specified -- on each platform it will different --
on platforms supporting themes or just preferences like allowing the
user to change the default font size, it might even change each time
depending on the configuration.  The actual size of the button is
computed automatically by the system at runtime when the button object
is created; the title is also translated automatically at runtime (and
the button sized depending on the translated title).  All this
information (appearance of buttons, fonts, images, translation of
titles etc) really is not known till the program is run, so the button
object will result different on each platform and each configuration.

For this reason, the list of objects really is more an abstract
description of the logic of the GUI interface rather than an exact
specification.  Most tags correspond to objects; a few tags (such as
the \texttt{<separator/>} tag, used to add separators in menus and
hboxes/vboxes) don't correspond to objects.

In some cases, different tags produce objects of the same class; for
example the \texttt{<label>} tag creates a \texttt{NSTextField} object
used to display a non-editable string (such as a title or a message),
while the \texttt{<textField>} tag creates a \texttt{NSTextField}
object used to display an area on screen where the user can insert
some text.  While they are implemented using the same class, they are
conceptually completely different GUI objects, because they have a
completely different function.

In other cases, the same tag could produce objects of different
classes depending on the attributes it contains.  I have no example at
hand but nothing prevents this to be done.

Normally, each tag creates an object, except in special cases, which
are clearly documented.

\subsection{Object tags - basic example}
Consider for example the \texttt{<window>} tag, used to create a
window object.  In the simplest case, the tag is just opened and
closed:
\begin{verbatim}
<?xml version="1.0"?>
<!DOCTYPE gsmarkup>

<gsmarkup>

  <objects>
    <window>
    </window>
  </objects>

</gsmarkup>
\end{verbatim}
or more simply (using the XML abbreviation of writing
\texttt{<window/>} meaning \texttt{<window></window>}):
\begin{verbatim}
<?xml version="1.0"?> 
<!DOCTYPE gsmarkup>

<gsmarkup>

  <objects>
    <window/>
  </objects>

</gsmarkup>
\end{verbatim}
(in this example, we showed all the document; in the next examples of
object tags, we will omit all but the tags we are discussing).  If you
load this example into an application, it will create a window, with
no particular attributes and nothing inside it.

\subsection{Object tags - attributes}
If you want to modify the properties of an object, you add attributes
to the tag.  For example, if you create a window using the
\texttt{<window>} tag, it is created by default miniaturizable, resizable and 
closable.  If you don't want it to be closable, you need to add an attribute
to the tag -
\begin{verbatim}
<window closable="no"/>
\end{verbatim}
This causes the window to be created without the close button.
Attributes are generally used to change the properties of that
specific object.  Each object is normally created with default
properties set up for the most common case, so you only need to
specify attributes when you need some special behaviour.  In the
example of the window, all windows are created by default with a close
button, unless you specify \texttt{closable="no"}.

A tag can have more than one attribute -- it can have as many
attributes as you need.  The attributes are written one after the
other one, separated by spaces.  For example, if you want both to specify
that a window should not be closable and not resizable, you can write
\begin{verbatim}
<window closable="no" resizable="no"/>
\end{verbatim}

We now examine and classify all the different types of attributes you
can use for a tag.  Some attributes have a string value, other a
boolean value, other even describe a color or a font; other still give
a name to the object so that it can be referenced from elsewhere;
other are references to other objects.

\subsubsection{String attributes}
String attributes are very common -- a string attribute sets a
property of the object which is described by a string.  For example,
if you want to create a window with a title \texttt{Game Board}, you
write
\begin{verbatim}
<window title="Game Board"/>
\end{verbatim}
The attribute \texttt{title} is set to a string, \texttt{Game Board}
in the example.  You can use any character in the string, except a few
special XML characters -- to insert them, you need to use the entities
listed in table \ref{XML-entities}.  Each tag has a different set of
available string attributes -- the reference documentation later on in
this document lists all available string attributes for each tag.

\subsubsection{Localizable strings attributes}
Most (but not all) strings need to be localized -- that is, translated
into the user preferred language -- before they are displayed.  Each
tag knows which attributes are strings to localize, and which are
strings not to localize (it's normally quite obvious -- anything which
is meant to be displayed to the user must be localized, while anything
which is only used internally by the program must not be; for example,
window titles or button titles are localized; selectors as used in
actions are not).  To localize a string, the gsmarkup framework calls the
NSBundle method +localizedStringForKey:value:table:, first looking for
the string in the specific table (normally loaded from a file having
the same name as the .gsmarkup file, but with a .strings extension rather
than a .gsmarkup extension), and failing that, in the Localizable.strings
strings table.  If that fails too, the string is used untranslated.
For example, suppose that you have the window
\begin{verbatim}
<window title="Game Board"/>
\end{verbatim}
in your \texttt{main.gsmarkup} file.  To translate it into Italian, you
add a file \texttt{Italian.lproj/main.strings} to your application,
containing the following line:
\begin{verbatim}
"Game Board"="Tavolo di Gioco";
\end{verbatim}
when the application loads the .gsmarkup file, it automatically translates
it into the appropriate language by using the appropriate
\texttt{main.strings} file for that language.  In this example, when the
user sets Italian as his preferred language, the window will be
created with title \texttt{Tavolo di Gioco} rather than \texttt{Game
Board}.  Please note that by default different .gsmarkup files use
different .strings files -- for each .gsmarkup file there is a
corresponding .strings file.  This makes it easier to keep the
.strings files updated (you only need to update a .strings file when
the corresponding .gsmarkup files has changed), makes clear where each
string comes from, and allows different .gsmarkup files to contain the
same English string, and yet to translate it in different ways.

\subsubsection{Boolean attributes}
Boolean attributes are used to set a boolean property of the object.
The valid values of a boolean attributes are \texttt{yes} and
\texttt{no} -- for example, you can write
\begin{verbatim}
<window resizable="no"/>
\end{verbatim}
Resizable is a boolean attribute, which can be \texttt{YES} or
\texttt{NO}.  The example sets it to \texttt{NO}.  For consistency 
with other stuff which is always written lowercase, we prefer to use
\texttt{yes} and \texttt{no} (lowercase), but it is only a matter of taste: 
\texttt{Yes}, \texttt{YES} are valid as well as attribute values, and they
mean the same thing as \texttt{yes}; \texttt{No} and \texttt{NO} are
valid as well to mean \texttt{no}.  For example, it is perfectly
correct to write
\begin{verbatim}
<window resizable="NO"/>
\end{verbatim}
and it means the same as before.

Each boolean attribute normally has a default value (documented in the
documentation); if a boolean attribute is not present, or if its value
can't be parsed as a \texttt{yes} or a \texttt{no}, then it's ignored,
which implicitly means that the default for it is used.

\subsubsection{Number attributes}
Number attributes are used to set a property to a number (integer or
float).  For example, to create window with width \texttt{200} and 
heigth \texttt{50}, you do
\begin{verbatim}
<window width="200" height="50"/>
\end{verbatim}

\subsubsection{Image attributes}
Image attributes are used to set a property to an image.  The image is
found by invoking the method \texttt{+imageNamed:} of
\texttt{NSImage}.  For example,
\begin{verbatim}
<button image="mail" title="Read Mail" imagePosition="above"/>
\end{verbatim}
would set the button image to the result of calling 
\begin{verbatim}
[NSImage imageNamed: @"mail"]
\end{verbatim}
and position it above the title.

\subsubsection{Color attributes}
Color attributes are used to set a property to a color.  You should
use this very rarely, because you want your program to use the native
platform colors when running on a certain platform.  But in some cases
-- for example when setting up a colorWell (a gui object which allows
the user to choose a color) -- you need to specify colors.

Color attributes are evaluated in the following way by the system: 
\begin{enumerate}
\item First, the system checks if \texttt{NSColor} responds to the method
\texttt{+xxxColor}, where \texttt{xxx} is the value of the attribute.  If it 
does, the method is invoked, and the result is used as the color.  For
example, 
\begin{verbatim}
<colorWell width="50" height="50" color="red"/>
\end{verbatim}
will create a colorWell which displays the color returned by
\texttt{[NSColor redColor]}.  In this way, you can access all standard 
colors by name, such as \texttt{black}, \texttt{white}, \texttt{red},
\texttt{blue}, \texttt{green}, \texttt{yellow}, \texttt{windowBackground}, etc.
(advanced tip: if you want to access a non-standard one by name, you
can always implement a \texttt{NSColor} category, and add the method
you need).
\item If the method is not found, the system tries to parse the color as 
a calibrated RRGGBB or RRGGBBAA description of the color, where RR,
GG, BB and AA are hexadecimal two-digits number between 00 and FF
describing the red, green, blue and alpha components of the color (btw
this is the same format used in HTML pages, without the \texttt{\#} at
the beginning).  For example, \texttt{FFFFFF} is white,
\texttt{000000} is black, \texttt{FF0000} is red, and \texttt{FF000065} 
is a semi-transparent red color.
\end{enumerate}

\subsubsection{Font attributes}
Font attributes are used to set a font property.  This is very
delicate, because you want your program to use the native platform
fonts when running on a certain platform -- hardcoding a certain font
might actually destroy the visual impact of your program rather than
improving it (typically, the font you want might not be available on
that system, and be replaced with an ugly one, or might simply clash
completely with the rest of the look on the system, be too big, too
little, or too bold or too fuzzy compared with the other native fonts
used by the system).  In general, system fonts are normally chosen
between the prettier fonts available on the plaform, and the
appearance of the user interface elements is made to match those
fonts.  Last, the user/theme might have changed the fonts to fit their
wishes/needs, and you certainly don't want to argue with your users.
For this reason, gsmarkup normally doesn't really allow you to specify a
font, but only to scale the default system fonts.

In some cases -- for example when creating panels -- you might want to
display a title on top of your panel, or other information which you
want to display in a more prominent font.  This is typically needed by
labels (objects which display non-editable strings such as those used
by titles or messages).  A font attribute allows you to specify the
size and type of font you want.

The font for labels is normally found by calling the \texttt{[NSFont
labelFontOfSize: 0]} method (or the \texttt{[NSFont systemFontOfSize:
0]} on older systems which don't have \texttt{+labelFontOfSize:}).

You normally only need to make the font bigger or smaller.  To do it,
you can use the following font values:
\begin{itemize}
\item {\bf Tiny}: uses the same font, reduced to 1/3 of the original size.
\item {\bf tiny}: uses the same font, reduced to 1/2 of the original size.
\item {\bf Small}: uses the same font, reduced to 2/3 of the original size.
\item {\bf small}: uses the same font, reduced to 4/5 of the original size.
\item {\bf medium}: uses the same font, with no size change.
\item {\bf big}: uses the same font, enlarged 5/4 times the original size.
\item {\bf Big}: uses the same font, enlarged 3/2 times the original size.
\item {\bf huge}: uses the same font, enlarged 2 times the original size.
\item {\bf Huge}: uses the same font, enlarged  3 times the original size.
\end{itemize}
For example,
\begin{verbatim}
<label font="Big">Contents</label>
\end{verbatim}
displays \texttt{Contents} using the default label font, with the
default label font size multiplied by 1.5.

You can also specify a float, which is read and interpreted as a
scaling factor.  For example, 
\begin{verbatim}
<label font="1.5">Contents</label>
\end{verbatim}
is completely equivalent to setting the font to \texttt{Big}.

Normally, you only want to change the fonts in labels, and you don't
need to bother with the font type.  For very special cases, you might
need to change font type.  The current API used on OpenStep systems
provides NSFont methods to get recommended fonts to be used on that
particular platform to display various types of gui objects -- for
example, \texttt{[NSFont userFontOfSize: 0]} returns the font to be
used for user editable text, in the default size.  To choose that type
of font, you just specify \texttt{user} as the font value, as in
\begin{verbatim}
<label font="user">Testing fonts</label>
\end{verbatim}
this draws the label using the font \texttt{[NSFont userFontOfSize:
0]}.  In general, if the value of the font attribute is xxx, the
NSFont method \texttt{+xxxFontOfSize:} (if it exists) is used to get
the font.  The types of fonts available on gsmarkup at the moment are
\texttt{label}, \texttt{boldSystem}, \texttt{user}, 
\texttt{userFixedPitch}, \texttt{menu}, \texttt{message},
\texttt{palette}, \texttt{system}, \texttt{titleBar},
\texttt{toolTips}.  These fonts should be available on other OpenStep
systems as well, but not all them are available on older OpenStep
systems, and you would extremely rarely need to use them anyway.

Anyway, if for example you want to display a button containing a title
with a bigger font, you may need them.

You can use one of those standard fonts in any size, for example
\begin{verbatim}
<label font="user big">Testing fonts</label>
\end{verbatim}
is valid, and uses the same font returned by \texttt{[NSFont
userFontOfSize: 0]} but enlarged by a factor of 1.20.

\subsubsection{The id attribute}
This is a special attribute which any object can have; it is used to
tag objects with names, so that you can refer to that object by name.
The attribute is always called \texttt{id}, and its value is a string
which is the internal name to give to the object\footnote{The string
should not contain any dots (that is, it should not contain the
\texttt{'.'} character), because dots are used for advanced key value
coding.}.  For example,
\begin{verbatim}
<window id="board"/>
\end{verbatim}
creates a window, and tags it with the id \texttt{board}.  You can
then refer to this window object by name whenever you need it (we'll
explain later on when you might need to refer to the window object).
Of course, you don't need to tag all objects with names -- you would
just be wasting time and making your gsmarkup files more difficult to read
(and slower to parse and load by applications) if you do -- you only
need to tag the ones which you need to refer to.  The id name of the
window is never shown to the user -- it's only used internally to
refer to objects by name, and establish connections between objects.

\subsubsection{Attributes referring to other objects}
In the previous section we have explained how you can tag an object
with an id.  In this section we learn how you can refer to an object
by id.  There is a very general syntax which can be used to refer to
an object which has been given an id: you prefix the id name with a
hashmark (\texttt{\#}).  For example, if you create a window and give
it an id of \texttt{board}, as in
\begin{verbatim}
<window id="board"/>
\end{verbatim}
then you can refer to that window object by using the syntax
\texttt{\#board}.  \texttt{\#board} simply means {\it the object whose id name 
is} \texttt{board}.

This is normally used to establish connections between objects.  There
are two ways of building connections between objects: one is adding a
connector to a connectors section.  A gsmarkup file can contain one or
more connectors sections, where you can freely build up connections
between objects (and you refer to objects by using their ids and the
syntax just explained).  The other way, the one we learn about here,
is by using attributes which take objects as values.

Some attributes wants other objects as values.  For example, a window
can have a delegate, and this delegate is an object.  You can specify
the delegate object for a window by using the following syntax:
\begin{verbatim}
<window delegate="#myDelegate">
\end{verbatim}
When this file is loaded in a running application, the attribute 
\texttt{delegate} is treated specially.  It is treated specially because
its value begins with a hash (\texttt{\#}), which means it is a
reference to another object (in the file or even outside the file).
When the file is loaded, the method \texttt{setDelegate:} of the
window object will automatically be called, with argument the object
which has id name \texttt{myDelegate}.  This object can be 
\begin{enumerate}
\item another object which is created by the file when the file is loaded. 
     In the example, since you can create objects of custom classes in
     the file, you could allocate an instance of a custom class, give
     it the id name \texttt{myDelegate}, and then make it the window
     controller.

\item the file owner.  The file owner is an existing object in the application 
     which loads the file; whenever the application loads a gsmarkup file,
     it needs to specify a file owner object, which is automatically
     made available to the file with the id name \texttt{NSOwner}.
     For example, to make the file owner the delegate of your window,
     you would do
\begin{verbatim}
<window delegate="#NSOwner">
\end{verbatim}
     Normally, all connections between the objects loaded in the file
     and the existing application objects are done through the file
     owner.

\item NSApp (the shared application instance), which is automatically 
     made available by Renaissance with the id name \texttt{NSApp}.
     For example, to create a popup button with an item which
     terminates the application, you would do
\begin{verbatim}
<popUpButton>
  <popUpButtonItem title="terminate the app" action="terminate:"
                   target="#NSApp" />
</popUpButton>
\end{verbatim}

\item some other existing object in the application which is made available 
     to the file when the file is loaded.  If the application needs
     it, by using the extended NSBundle GSMarkup API, it can make
     available any existing object in the application specifying under
     which name it should be available.  If there is a single object
     to connect, it is simpler to make that object available as file
     owner; if there are more, you can specify them in the external
     name table dictionary (see description of the NSBundle GSMarkup
     API), and then the file can make any connections to those objects
     by referring to them by id, and using the special syntax with the
     hash.

\end{enumerate}

Here is another example --
\begin{verbatim}
<window>
  <hbox>
    <button title="yes" id="yes" nextKeyView="#no"/>
    <button title="no" id="no" nextKeyView="#yes"/>
  </hbox>
</window>
\end{verbatim}
in this example, a window with two buttons is created.  The first
button has title \texttt{yes}, the second one has title \texttt{no}.
The first button is given id \texttt{yes}, and the second one is given
id \texttt{no}.  Then, the nextKeyView attribute of the first button
is set to point to the second one, and the nextKeyView attribute of
the second button is set to point to the first one.  This allows users
to navigate between the buttons using the keyboard -- pressing TAB
while the input focus is on an object moves the focus to the
nextKeyView of that object.  In the example, the nextKeyView of the
first button is the second one, and viceversa, so by pressing TAB the
user can move the input focus between the two buttons.

\subsubsection{Technical details of embedded outlets}
You can skip this section at a first reading.  It is explaining the
technical details of attributes referring to objects.

Technically, whenever an attribute has a value which begins with a
hash (\texttt{\#}), the system automatically creates an outlet
connector when the file is loaded (and adds it to the list of outlet
connections to be established), and removes the attribute from the
list of attributes.  No check is done on the attribute name, so you
are free to setup any sort of connectors you want.  When the outlet
connections are established, the outlet will establish the connection
from the source (the object bound to the tag where the attribute was
found) to the target (the object which is referred by the attribute)
using as key the attribute name.  Establishing outlet connections uses
key-value coding, so in practice by writing
\begin{verbatim}
<window delegate="#myDelegate">
\end{verbatim}
you are effectively establishing that the value of the key
\texttt{delegate} of the created window object should be set to the
object whose id is \texttt{myDelegate}.

This way of embedding outlets inside object description is both very
natural and very powerful; it is certainly much more natural than
having to write in the separated connectors section.

\subsubsection{Advanced embedded outlets using key value paths}
Renaissance also allows the advanced syntax
\begin{verbatim}
<window delegate="#NSOwner.windowDelegate"/>
\end{verbatim}
to mean setting as delegate the result of calling \texttt{[\#NSOwner
valueForKeyPath: @"windowDelegate"]}.  Generally, if the id of the
source or target of an embedded outlet (or other connector) contains a
dot (that is, the character \texttt{'.'}), then the string up to the
dot is the id of an object to use, and the string following the dot is
interpreted as a key path value to apply to the object in order to get
the actual source or target of the connector.

For example, \texttt{\#NSApp.mainMenu} means the result of calling
\texttt{[NSApp valueForKeyPath: @"mainMenu"]} (which will return
the application main menu).  \texttt{\#NSApp.delegate.description}
means the result of calling \texttt{[NSApp valueForKeyPath:
@"delegate.description"]} (which will get the \texttt{NSApp} delegate,
then call \texttt{description} on it).

Please refer to the documentation of the \texttt{-valueForKeyPath:}
method for more information.

\subsection{Object tags - content}
Everything which is found between an open tag and the corresponding
close tag is called the content of the tag.  For example, in
\begin{verbatim}
<label>Please delete all files from your hard disk immediately</label>
\end{verbatim}
the open tag is \texttt{<label>}, the close tag is \texttt{</label>},
and the string inside is the content of the tag.  Often, the content
of a tag can include other tags.  For example,
\begin{verbatim}
<window>
  <button title="Click me to quit" action="terminate:"/>
</window>
\end{verbatim}
in this case, the content of the window tag is the button tag.

In general, while attributes are used to change the properties of the
specific objects represented by that tag, the content of a tag are
used to describe things which go {\sl inside} the object.  This
concept is of course human and not completely well defined, anyway the
idea is that, for example, the attributes of a \texttt{<window>} tag
(such as \texttt{closable="no"}, \texttt{width="200"},
\texttt{title="Main Window"}) change the general properties of the
window, while the tags inside the \texttt{window} tag represent
objects which should be displayed inside the window.

Now we examine typical examples of tags and the type of content they
can hold.

\subsubsection{No content}
Some gui objects do not contain anything.  For example, a button or a
colorWell do not contain anything and never have a content.

\subsubsection{String content}
Some gui objects have a simple string content, representing a string
to be displayed inside the gui object.  This typically happens with
labels and textfields; in this case, the string is not in the
attributes because it could potentially be very large.  For example,
\begin{verbatim}
<label>Missing image</label>
\end{verbatim}
displays a label with the string \texttt{Missing image}, while
\begin{verbatim}
<textField>user@host</textField>
\end{verbatim}
displays an editable textField which contains the string
\texttt{user@host} when it is first displayed.  In this case, the string 
to display is in the content of the tag because it can potentially be
very big (and perhaps optionally contain formatting tags (html?) in
the future for attributed strings ?).

In most cases, string content are localizable -- they are
automatically localized by the gsmarkup framework in the same way as it is
done for localizable string attributes.

\subsubsection{Tag content}
Other gui objects have content which is composed of other objects.
Typical examples include:
\begin{itemize}
\item Menus.  A menu contains menu items.  Here is an example:
\begin{verbatim}
<menu type="main">
  <menuItem title="Info" action="showStandardInfoPanel:" key="i"\>
  <menuItem title="Hide" action="hide:" key="h"\>
  <menuItem title="Quit" action="terminate:" key="q"\>
</menu>
\end{verbatim}
When the menu is created, the menuitems are created and added to it.
A menuitem can include a (single) menu -- this happens when clicking
on the menuItem opens a submenu.
\item Windows.  A window can contain a single object, which is set to be
the window's contentview.  The view can contain other objects, so 
building the whole view tree inside the window.
\item Boxes.  Horizontal, vertical boxes contain a list of objects, and display
all these objects in a row (in the case of horizontal boxes), or in a column
(in the case of vertical boxes).  For example,
\begin{verbatim}
<hbox>
  <button title="Save"/>
  <button title="Discard"/>
  <button title="Cancel"/>
</hbox>
\end{verbatim}
displays the buttons inside the horizontal box (which is invisible, it
is only used to align the buttons).  Simple boxes contain a single
object, which they display surrounded by a border (and optionally with
a title).  For example,
\begin{verbatim}
<box title="Mail Preferences">
  <vbox>
    <!-- missing - vbox content -->
  </vbox>
</box>
\end{verbatim}
displays a border box with title \texttt{Mail Preferences}, and inside it 
a vertical (invisible) box containing any sort of objects (not shown in the
example).
\item Containers (boxes are actually just a special case of containers), such
as a scrollview.  A scrollview contains an object, which is the
scrollview document view.
\item Tables.  A table contains various columns.
\end{itemize}

\section{The connectors section}

The connectors section is used to establish connections between
objects.

\subsection{How objects are given ids}
Objects are identified by their \texttt{id}.  There are various ways
in which an object can be given an \texttt{id}:
\begin{itemize}
\item Objects created by the file are given an id if the tags which creates
them contains the \texttt{id} attribute.  For example,
\begin{verbatim}
<textField id="pop3" width="200"/>
\end{verbatim}
the textField object is given the \texttt{id} \texttt{pop3}.
\item The file owner -- an object which is specified by the code loading the
GSMarkup file, and normally is the main way of connecting objects in the
GSMarkup file with the rest of the application -- is automatically given
the \texttt{id} \texttt{NSOwner}.
\item Additional objects which the code loading the GSMarkup file wants to make
available to the GSMarkup file.  By specifying an external name table when
loading the GSMarkup file, the code can give an \texttt{id} to any
existing object in the application (and so making it available for
establishing connections to/from it).
\item Some predefined objects have a predefined \texttt{id} given to them
automatically by the system.  For example, the main
\texttt{NSApplication} instance is automatically available with \texttt{id}
\texttt{NSApp}.
\end{itemize}
Any object which has been given an id can be the source or the target
of a connection set up in the connectors section.

\subsection{Connector tags}
Inside the connectors section tags are interpreted in a different way
than inside the objects section.  Inside the connectors sections, tags
are interpreted as being connectors tags rather than object tags.
Connector tags might vary, but normally you only need to know about
outlet connectors, so we only describe outlet connectors here.

\subsection{Outlet connectors}
An outlet connector is created with the \texttt{<outlet>} tag.  For
example,
\begin{verbatim}
<connectors>
 <outlet source="#NSOwner" target="#Controller" key="controller"/>
</connectors>
\end{verbatim}
An outlet has a source (an object specified by id), a target (another
object again specified by id), and a key (a string).  The outlet
describes a connection which should be established between the source
and target object; the key represents a key in the source object.  By
using key-value coding, when the outlet connection is established, the
value for the key in the source is set to the target (I'll try to
explain this better in a minute).  In the example, when the outlet is
established, it's result will likely to be to invoke
\begin{verbatim}
[#NSOwner setController: #Controller];
\end{verbatim}
(where \texttt{\#NSOwner} is the object with id \texttt{NSOwner} and 
\texttt{\#Controller} is the object with id \texttt{Controller}).

As an advanced extension, if the \texttt{source} or \texttt{target} of
an outlet connector contains a dot (that is, the \texttt{'.'} 
character), then the part of the string before the dot is interpreted
as an object specified by id, and the part of the string after the dot
is interpreted as a key value path to use on the object to get the
actual \texttt{source} or \texttt{target}.  For example, when the outlet
\begin{verbatim}
<connectors>
 <outlet source="#NSOwner" target="#Controller.name" key="controllerName"/>
</connectors>
\end{verbatim}
is established, it's result will likely to be to invoke
\begin{verbatim}
[#NSOwner setControllerName: [#Controller name]];
\end{verbatim}
assuming that \texttt{\#Controller} responds to a method
\texttt{name} (if it doesn't, key-value coding -- as explained 
in the next section -- is used to retrieve the value for key
\texttt{name} of the object \texttt{\#Controller}).

\subsection{Details of how outlet connections are established}
When an outlet connection is established, the following things happen:
\begin{enumerate}
\item the outlet connector determines the source and target objects 
from their ids (by performing a lookup in the name tables which store
the mappings from ids to objects).  If the ids do not contains dots, a
simple lookup in the name table is done.  If the source or target
strings contain a dot, then the advanced key value coding support is
used: the string is broken in two parts: before and after the first
dot in the string.  For example, \texttt{\#NSController.delegate.name}
would be broken into \texttt{\#NSController} and
\texttt{delegate.name}.  The first part of the string is looked up in
the name table, producing an object.  Then, key value coding is used
to retrieve the value for the key path given by the second string on
this object.  In practice, if the first part of the string is
\texttt{xxx}, and the second part is \texttt{yyy}, the
\texttt{source} or \texttt{target} is set to the result of the following
operations:
\begin{verbatim}
[[nameTable objectForKey: xxx] valueForKeyPath: yyy];
\end{verbatim}
(please check the documentation for \texttt{valueForKeyPath:} to learn
more about this operation).
\item the outlet connector calls
\begin{verbatim}
[source takeValue: target forKey: key];
\end{verbatim}
to set target as the value for the key key in the source.  If you
don't know what this really mean, we describe here briefly what this
does; for more information please read the documentation of the
\texttt{takeValue:forKey:} method.
\begin{enumerate}
\item if the key is xxx, the system searches for the \texttt{setXxx:}
method in the source (please note that the first letter of xxx is made
uppercase).  If that method exists, the following code is executed:
\begin{verbatim}
[source setXxx: target];
\end{verbatim}
and the outlet connection can be considered established.
\item if the method wasn't found, the \texttt{\_setXxx:} method is searched,
and if it exists is used.
\item if this fails, the system searches an instance variable with name
\texttt{\_xxx} in the source, and if it finds it, it \texttt{ASSIGN}s to it
the target\footnote{Unless the class of the source has implemented
\texttt{+accessInstanceVariablesDirectly} to return \texttt{NO}, in
which case the system skips trying to assign instance variables in
this way.}.  Assigning includes releasing the previous value, and
retaining the new one.
\item if the \texttt{\_xxx} instance variable is not found, the system
searches for an instance variable called \texttt{xxx}, and tries to set
that.
\item if even this fails, the system calls the method
\texttt{-handleTakeValue:forUnboundKey:} of the source, which -- unless
overridden by the source class -- raises an exception.
\end{enumerate}
\end{enumerate}
As you can see, you can automatically set up instance variables of
objects to point to other objects when you load a gsmarkup file.  This is
very handy to establish connections between objects.

\section{The GSMarkup NSBundle Additions API}

\subsection{What is the NSBundle Additions API}
The main API for loading GSMarkup files into a running application is via
the NSBundle GSMarkup Additions.  These are a set of category methods of
NSBundle which allow you to load a gsmarkup file into a running
application, or to get a list of the localizable strings in a gsmarkup
file (this is useful for applications which allow you to manage gsmarkup
files).  The methods are declared in
\texttt{Markup/GSMarkupBundleAdditions.h}, so you need to \texttt{\#include} 
(or \texttt{\#import}) that file to use the methods.  Please note that
the API mirrors the existing NIB loading API which is traditionally
found on OpenStep.

\subsection{Methods reference}
We now list each method, followed by the explanation of what the
method does.

\subsubsection{NSBundle methods}
The methods listed in this section are methods of NSBundle.
\begin{verbatim}
+ (BOOL)   loadGSMarkupFile: (NSString *)fileName
      externalNameTable: (NSDictionary *)context
               withZone: (NSZone *)zone
localizableStringsTable: (NSString *)table
               inBundle: (NSBundle *)localizableStringsTableBundle;
\end{verbatim}
This method is the method which does the actual loading; other loading
methods end up calling this one to do the loading.  This method loads
the file \texttt{fileName} (an absolute path) into the application.
If \texttt{fileName} is missing the \texttt{.gsmarkup} extension, it is
automatically appended.  The file is read, and all the sections are
parsed, resulting in a list of objects and of connectors, and a name
table mapping id names to objects which have been decoded.  The
objects are then asked to replace themselves with the corresponding
platform objects, and the name table is updated accordingly.  While
replacing themselves with the platform objects, the objects also
translate the eventual text which might need to be translated by using
the localizable strings given as argument, which is found in the
bundle given as argument.  Each type of objects knows which attributes
or which content requires translation and which don't.  Then, the
\texttt{context} dictionary (which is supposed to contain a mapping of
strings -- id names -- to objects already existing in the application;
typically it contains at least the \texttt{NSOwner} object) is merged
into the name table loaded from the file (except the special key-value
pair with key \texttt{NSTopLevelObjects}, which is not added to the
name table, and it is used later on for another task).  The
connections are finally established using the final name table.  Last,
all top-level objects which were decoded from the file are sent the
\texttt{awakeFromGSMarkup} method.  The top-level objects so created are
not autoreleased -- so unless they are released at some point, they
will never be deallocated.  This is correct for example for windows,
which are normally released when the user closes them; it is also
correct for some custom objects (the main controller object in the
application), which are supposed to just exist for the whole lifetime
of the application.  But there are cases in which you might need
access to the top-level objects, for example in order to release them.
This is done in a similar way as it is done in NSBundleAdditions,
where an undocumented Apple extension can be used to get the top-level
objects.  If the \texttt{context} dictionary contains a key
\texttt{NSTopLevelObjects}, with value a NSMutableArray, this array is
filled with the top-level objects created from the .gsmarkup file when it
is read.  \texttt{zone} is supposed to be the memory zone from which
all objects created when reading the file are allocated; but it is
currently ignored, so you can pass \texttt{NULL} to it.  The strings
table and bundle arguments are used to translate the messages
contained in the objects; if table is nil, the path extension is
removed from the filename, and that is used as string table (so that
localizable strings for the gsmarkup file \texttt{example.gsmarkup} are by
default searched in the localizable strings file
\texttt{example.strings}).  If bundle is nil, the main bundle is used.
The method returns \texttt{YES} if the file could be loaded, and
\texttt{NO} otherwise.

\begin{verbatim}
+ (BOOL) loadGSMarkupFile: (NSString *)fileName
    externalNameTable: (NSDictionary *)context
	     withZone: (NSZone *)zone;
\end{verbatim}
This method is a short form of the previous one; it simply calls it
with nil table and nil bundle, which means that the localizable
strings are translated using the strings table with the same name as
the .gsmarkup file (extension removed) in the main bundle.

\begin{verbatim}
- (BOOL)    loadGSMarkupFile: (NSString *)fileName
       externalNameTable: (NSDictionary *)context
                withZone: (NSZone *)zone
 localizableStringsTable: (NSString *)table;
\end{verbatim}
This method loads the gsmarkup file with name \texttt{fileName} (NB: if
the \texttt{fileName} string does not have the extension .gsmarkup, it is
automatically added) from the receiver bundle.  The method first
locates the file to load in the bundle, by searching as in the
following example:
\begin{verbatim}
bundle_path/Resources/Italian.lproj/fileName.gsmarkup
bundle_path/Resources/English.lproj/fileName.gsmarkup
bundle_path/Resources/fileName.gsmarkup
bundle_path/Italian.lproj/fileName.gsmarkup
bundle_path/English.lproj/fileName.gsmarkup
bundle_path/fileName.gsmarkup
\end{verbatim}
assuming that \texttt{Italian.lproj} is the user's preferred language.
This algorithm differs from the standard bundle searching algorithm in
that localized resources are preferred to non-localized ones (we're
still all wondering why NSBundle does not have a standard method doing
this by default).  Once the file has been located and its absolute
path on disk is known, the method finally calls the NSBundle method
% FIXME - the following line shows up incorrectly on HTML - the
% convertor needs to be fixed!
\texttt{+loadGSMarkupFile\discretionary{:}{}{:}externalNameTable\discretionary{:}{}{:}withZone\discretionary{:}{}{:}localizableStringsTable\discretionary{:}{}{:}inBundle:} 
with the given table as table argument, and the receiver bundle as
bundle argument, to perform the actual loading.

\begin{verbatim}
- (BOOL)    loadGSMarkupFile: (NSString *)fileName
       externalNameTable: (NSDictionary *)context
                withZone: (NSZone *)zone;
\end{verbatim}
This is a short form of the previous method, which uses a nil
localizableStringsTable, which causes the localizable strings file
with the same name as the gsmarkup file (extension removed) to be used as
localizable strings table.

\begin{verbatim}
+ (BOOL) loadGSMarkupNamed: (NSString *)fileName
                 owner: (id)owner;
\end{verbatim}
This method is the more straightforward API to load a gsmarkup file.
\texttt{fileName} should be a file name (including the .gsmarkup
extension or not) (not an absolute path, just a file name);
\texttt{owner} should be an object provided by the application.  Both
must not be nil, otherwise the method immediately returns \texttt{NO}.
The method first builds a context dictionary containing a single
key-value pair, which maps the string \texttt{NSOwner} to the object
\texttt{owner}; then, it gets the owner bundle (if the owner object
belongs to a bundle), or the main bundle (if the owner object does not
belong to a bundle); finally, it invokes the method
\texttt{-loadGSMarkupFile:externalNameTable:withZone:} of that bundle to load
the gsmarkup file.  It uses as zone the owner's zone (but this detail is
currently ignored).  It returns the result of invoking that method.
Localizable strings are by default translated using a table with the
same name as the gsmarkup file (extension removed, so that the localizable
strings file for \texttt{test.gsmarkup} would be \texttt{test.strings}),
from the bundle in which the file is loaded.

\begin{verbatim}
+ (NSArray *) localizableStringsInGSMarkupFile: (NSString *)fileName;
\end{verbatim}
This method parses the \texttt{fileName} file (which should be an
absolute path to a file on disk; if the \texttt{fileName} string does
not end with \texttt{.gsmarkup}, this extension is automatically added),
and it extracts the list of localizable strings in the file.  Each tag
knows exactly which strings found in its attributes and contents are
localizable; the list of localizable strings is built using this
knowledge.

\subsubsection{awakeFromGSMarkup informal protocol}
This section contains a single method, which any object can implement
to perform additional setup after it has been created from a GSMarkup
file.
\begin{verbatim}
- (void) awakeFromGSMarkup
\end{verbatim}
This method is called in the top-level objects created from a GSMarkup
file after all connections (outlets etc) have been established.  You
can implement this method to complete the setup of your objects.

%%
%% Object tag reference
%%

\section{Object tag reference}

In this reference, we document all available tags in an objects
section.  Because new tags can be dynamically added by applications
and bundles, we only document the standard system tags.  The
description of each tag consists of
\begin{itemize}
\item The description of the tag, telling you what the tag is.
\item The list of valid attributes for the tag.  All attributes are optional 
(except in very rare cases which are carefully documented).  
The section tells you what values you can give to each attribute, and
what the attribute does.  The \texttt{id} attribute is always
available for all objects so it's not listed here.
\item The list of possible meaningful object attributes for the tag.
These are converted into outlet connectors at run-time.  The list
lists standard attributes which take a value which is another object
(identified by id).  Because of the way object attributes work (by
setting attributes at run-time using key-value coding), there might be
more attributes than the ones listed here.
\item A description of what content the tag accepts.
\item One or more example gsmarkup code using the tag.
\item In a few specific cases, a discussion of portability issues.
\end{itemize}

%
% Template for documenting new tags
%

%\subsection{The xxx tag}
%
%\subsubsection{Description}
%A \texttt{<xxx>} tag represents xxx, and normally generates an instance
%of \texttt{XXX}.
%
%\subsubsection{Attributes}
%\begin{itemize}
%\item {\bf ttt}:
%\end{itemize}
%
%\subsubsection{Possible object attributes}
%\begin{itemize}
%\item {\bf ttt}:
%\end{itemize}
%
%\subsubsection{Content}
%
%\subsubsection{Examples}
%
%%OPTIONAL%%
%\subsubsection{Portability}

\subsection{The image tag}

\subsubsection{Description}
An \texttt{<image>} tag represents an image, and generates an instance
of \texttt{NSImageView}.  An \texttt{NSImageView} is a control which
is used to display an image.  Normally, you specify the image by
putting an \texttt{name="xxx"} attribute into the tag, and the
\texttt{NSImageView} is automatically sized to fit the image.
The value of the \texttt{name} attribute is a string, which is used as
argument to the \texttt{NSImage} method \texttt{+imageNamed:}.
Typically you put the image (as a .png, or .gif, or .jpeg , or .tiff
file) in the main application bundle.

If all you want to do is display an image from a file, this is all you
need to know; but there are more complex situations which might
require more control: for example, you can change the image
programmatically later (by calling \texttt{-setImage:} of the
\texttt{NSImageView}), after the file has been loaded; or you can
force a size of the image view (by using the \texttt{width} and
\texttt{height} attributes which the \texttt{<image>} tag inherits 
from the generic \texttt{<view>} tag), which might be different from
the size of the image contained in the file; or you can set the image
view to be editable (by setting \texttt{editable="yes"}), and then the
user can change the image by dragging\&dropping images from elsewhere.

In all those advanced cases, you can easily end up with an image view
of a size which is different from the size of the image to be
displayed in the view.  Usually the image is then automatically
scaled, but you can decide how the scaling is to be done by using the
\texttt{scaling} attribute, and how to align the image in the area (if
not all of it is covered after the specified scaling) by using the
\texttt{alignment} attribute.  Please check the \texttt{NSImageView} 
documentation for more information on scaling and alignment.

\subsubsection{Attributes}
The \texttt{<image>} tag inherits all attributes from the generic
\texttt{<control>} and \texttt{<view>} tags; in addition, it has 
the following ones:
\begin{itemize}
\item {\bf alignment}: The alignment (if needed).  This attribute is 
a string, and is optional.  It can be set to \texttt{bottom},
\texttt{bottomLeft}, \texttt{bottomRight}, \texttt{center}, 
\texttt{left}, \texttt{right}, \texttt{top}, \texttt{topLeft} or
\texttt{topRight}.  If you specify a size for the \texttt{<image>} tag 
which is bigger than the size of the image to be displayed in the
image view, this specifies how the image is aligned after having been
scaled.
\item {\bf name}: The name name of the image to display.  This attribute
is a string, and it is optional, but generally you should have it
unless you are setting an image programmatically later.  If this
attribute is set, the image to display is found using \texttt{[NSImage
+imageNamed:]}, where the value of this attribute is used as the
argument; if this attribute is not set, no image is actually
displayed.
\item {\bf scaling}: The type of scaling (if needed).  This attribute 
is a string, and it is optional.  It can be set to \texttt{none},
\texttt{proportionally} or \texttt{toFit}.  It determines how the image
is scaled if the frame size for the image view is different from the
size of the image to display in the image view.
\end{itemize}

\subsubsection{Possible object attributes}
The \texttt{<image>} tag inherits all object attributes from the
generic \texttt{<control>} and \texttt{<view>} tags, but adds no
specific of its own.

\subsubsection{Content}
No content.

\subsubsection{Examples}
In the typical usage, you have an image in your main application
bundle called for example \texttt{GreenArrow.png}.  To display the
image in a window, you can use the \texttt{<image>} tag:
\begin{verbatim}
<image name="GreenArrow" />
\end{verbatim}

\subsection{The instance tag}

\subsubsection{Description}
An \texttt{<instance>} tag represents an instance of a custom class.  You
specify the custom class and an object of that class is created.

\subsubsection{Attributes}
\begin{itemize}
\item {\bf class}: The name of the class.  This attribute is a string, and 
is required.
\end{itemize}

\subsubsection{Possible object attributes}
This depends on the class -- each class has different object attributes.

\subsubsection{Content}
No content.

\subsubsection{Examples}
Normally, you set the id of your object, because you are referring to
it somewhere else.  You also might set additional object attributes to
connect the object to other objects.
\begin{verbatim}
<instance class="MyController" id="Controller" 
          textField="#TextField" />
\end{verbatim}
in this example, the class \texttt{MyController} should have the method
\texttt{setTextField:}.

\subsection{The menu tag}

\subsubsection{Description}
A \texttt{<menu>} tag represents a menu, and normally generates an
instance of \texttt{NSMenu}.

\subsubsection{Attributes}
\begin{itemize}
\item {\bf autoenablesItems}: A boolean: if set to \texttt{no}, disables 
autoenabling of its menu items (which is otherwise enabled by default).
\item {\bf title}: The title of the menu (a localizable string).
\item {\bf type}: If set to \texttt{main}, the menu is made the
application main menu (by invoking \texttt{setMainMenu:} of the shared
\texttt{NSApplication}).  If set to \texttt{windows}, the menu is made
the application windows menu (by invoking \texttt{setWindowsMenu:} of
the shared \texttt{NSApplication}).  If set to \texttt{services}, the
menu is made the application services menu (by invoking
\texttt{setServicesMenu:} of the shared \texttt{NSApplication}).  If
set to \texttt{font}, the menu is made the application font menu (by
invoking \texttt{setFontMenu:} of the shared \texttt{NSFontManager}).
If set to \texttt{apple}, the menu is made the application apple menu
(by invoking \texttt{setAppleMenu:} of the shared
\texttt{NSApplication}); if running on Apple OSX; it is ignored
otherwise.
\end{itemize}

\subsubsection{Possible object attributes}
None.

\subsubsection{Content}
A menu can contain \texttt{menuItem} and \texttt{menuSeparator} tags.
Each \texttt{menuItem} tag represents a menu item which is added to
the menu.  Each \texttt{menuSeparator} tag represents a menu separator
which is added to the menu.

\subsubsection{Examples}
Normally, the only attribute of menus you're likely to use is the 
\texttt{type} attribute, which is used to make a menu the main, windows, 
services or font menu.
\begin{verbatim}
<menu type="main">
  <menuItem title="Info Panel..." action="orderFrontStandardInfoPanel:" />
  <menuItem title="Hide" action="hide:" key="h" />
  <menuItem title="Quit" action="terminate:" key="q" />
</menu>
\end{verbatim}

\subsubsection{Portability}
Menu portability is quite tricky.  Because there are many differences
between how the application main menus are organized on different
platforms, and what the standard names/locations used in menus are, it
is common to keep the application main menu in a separate gsmarkup
file, and to have a different gsmarkup file for each different
platform; then loading the appropriate file depending on the platform
you are running on.

In order to have your code work well across different platforms, it is
also recommended that you load the main application menu gsmarkup file
inside your \texttt{main()} function, before calling
\texttt{NSApplicationMain()} (if you call it).  On Apple OSX it is
also quite important to set the type (main, apple ...)  of the menus.
The essential thing is that there should be a single \texttt{main}
menu, containing the \texttt{apple} submenu as its first item.  There
should be one, and only one, \texttt{apple} submenu.  The simplest
(and probably best) way to get it right without too much effort is by
doing what working examples are doing (such as the CurrencyConverter
example, located inside
\texttt{Examples/Applications/CurrencyConverter} in the Renaissance
distribution).

\subsection{The menuItem tag}

\subsubsection{Description}
A \texttt{<menuItem>} tag represents a menu item to be added to a menu,
and normally generates an instance of \texttt{NSMenuItem}.

\subsubsection{Attributes}
\begin{itemize}
\item {\bf action}: The action of the menuItem (a string, non localized, 
which is converted into a selector).
\item {\bf enabled}: A boolean; if \texttt{no}, the menuItem is disabled (it
is enabled by default).  If your menu autoenables its items, this is not
particularly useful.
\item {\bf image}: The image of the menuItem (an image).
\item {\bf key}: The keyEquivalent of the menuItem (normally a single character).
\item {\bf state}: The state of the menuItem (either \texttt{on}, or \texttt{off} 
or \texttt{mixed}).  Setting a state for the menuItem automatically
causes the menuItem to display its state (on/off) ... at least on
GNUstep it is so, and it is done using a tickmark.
\item {\bf tag}: The tag of the menuItem (an integer).
\item {\bf title}: The title of the menuItem (a localizable string).
\end{itemize}

\subsubsection{Possible object attributes}
\begin{itemize}
\item {\bf target}: The target of the menuItem.
\item {\bf representedObject}: The representedObject of the menuItem.
\end{itemize}

\subsubsection{Content}
A menuItem can contain a single \texttt{menu} tag.  The menu is added
as a submenu to the menuItem.  The action and key equivalent of a
menuItem with a submenu are ignored.  The submenu automatically uses
the menuItem title as its title.

\subsubsection{Examples}
In most cases, you just set title, action and key of menu items (in
some cases not even those).  The classical menu item is the Quit menu
item of the main menu:
\begin{verbatim}
<menuItem title="Quit" action="terminate:"/>
\end{verbatim}
Here is an example of a menuItem holding a submenu:
\begin{verbatim}
<menuItem title="Info">
  <menu>
    <menuItem title="Info Panel..." action="orderFrontStandardInfoPanel:" />
    <menuItem title="Preferences..." action="runPreferencesPanel:" />
    <menuItem title="Help..." action="orderFrontHelpPanel:" />
  </menu>
</menuItem>
\end{verbatim}
The submenu automatically uses the menuItem title (\texttt{Info}) as
its title.  Here is an example of a menuItem displaying an on/off tickmark
and changing state:
\begin{verbatim}
<menuItem title="Options">
  <menu>
    <menuItem title="Full Headers" state="on" action="fullHeadersChanged:" />
    <menuItem title="Hide spam" state="on" action="hideSpamChanged:" />
  </menu>
</menuItem>
\end{verbatim}
You need to implement the actions, and have them change the state of the
menuItem each time, to see the state change.

\subsection{The menuSeparator tag}

\subsubsection{Description}
A \texttt{<menuSeparator>} tag represents a separator in a menu, and
normally inserts into the menu a \texttt{[NSMenuItem separatorItem]}.
It is a very simple tag, with no attributes nor content; it is opened,
and immediately closed, as in
\begin{verbatim}
<menuSeparator />
\end{verbatim}
It can only be used inside the contents of a menu.  It is usually
represented by a \texttt{[NSMenuItem separatorItem]} object, but that
is not compulsory -- it can be mapped to whatever standard way of
inserting separators in NSMenus is used on that platform.

\subsubsection{Attributes}
A \texttt{menuSeparator} tag has no attributes.

\subsubsection{Possible object attributes}
A \texttt{menuSeparator} tag has no possible object attributes.

\subsubsection{Content}
A \texttt{menuSeparator} tag has no content.

\subsubsection{Examples}
A menuSeparator is often used on Apple OSX to visually separate and
group menu items in menus, as in the following example:
\begin{verbatim}
<menu type="main">

  <menuItem title="CurrencyConverter">
    <menu title="CurrencyConverter" type="apple">
      <menuItem title="About CurrencyConverter" action="orderFrontStandardAboutPanel:" />
      <menuSeparator />
       <menuItem title="Services">
         <menu title="Services" type="services" />
       </menuItem>
       <menuSeparator />
       <menuItem title="Hide CurrencyConverter" action="hide:" key="h" />
       <menuItem title="Hide Others" action="hideOtherApplications:" />
       <menuItem title="Show All" action="unhideAllApplications:" />
       <menuSeparator />
       <menuItem title="Quit CurrencyConverter" action="terminate:" key="q" />
    </menu>
  </menuItem>
</menu>
\end{verbatim}

\subsection{The panel tag}

\subsubsection{Description}
A \texttt{<panel>} tag represents a panel, and normally generates an
instance of \texttt{NSPanel}.  The panel tag inherits from the window
tag, so it inherits all attributes available for a window.  The only
different with a window tag is that it creates a \texttt{NSPanel}
instead of a \texttt{NSWindow} object, and that it has a few more
attributes, specific to panels.

\subsubsection{Attributes}
All the attributes recognized by the \texttt{<window>} tag, plus:
\begin{itemize}
\item {\bf becomesKeyOnlyIfNeeded}: A boolean.  If set to yes, sets the panel
to become key only if needed.
\item {\bf floating}: A boolean.  If set to yes, sets the panel
to be a floating panel.
\item {\bf worksWhenModal}: A boolean.  If set to yes, sets the panel
to work when modal.
\end{itemize}

\subsubsection{Possible object attributes}
The same possible object attributes as for the \texttt{<window>} tag.

\subsubsection{Content}
The same content as the \texttt{<window>} tag.

\subsubsection{Examples}
\begin{verbatim}
  <panel title="Inspector" autosaveName="Inspector" floating="yes">
    <vbox>
      <!-- all sort of goodies go in here -->
    </vbox>
  </panel>
\end{verbatim}

\subsection{The window tag}
\subsubsection{Description}
A \texttt{<window>} tag represents a window, and normally generates an
instance of \texttt{NSWindow}.  When the tag is processed, the window
is first created using the style mask determined by the attributes
borderless, closable, resizable, miniaturizable, titled.  If no
attribute is specified, the window is created with all decorations, so
that it is made closable, resizable, miniaturizable and titled.  Any
of these decorations can be removed by setting the corresponding
attribute to no.  Setting the attribute borderless to yes causes no
decoration to be used.  After the window has been created, the content
of the tag is processed; it is expected to contain a single tag (or
nothing); if that tag produces a NSView object, the object is set to
be the window content view, and the window itself is resized so that
its content size is exactly that view's size (this basically autosizes
the window to fit the contents).  After the window has been so sized,
the hardcoded frame attributes are processed -- these attributes are
x, y, width, heigth, contentWidth and contentHeight.  If any of these
attributes is set, it is used to resize/relocate the window, so that
hardcoded frame attribute override the autosizing size.  Then, the
window sets additional attributes such as title, minWidth, minHeigth,
maxWidth, maxHeigth, backgroundColor.  Then, if the attribute center
is set to yes, the window is centered on screen.  At this point, if an
autosaveName is specified for the window, the window sets up
autosaving of the frame under that name and, if a frame already exists
in the user defaults with that name, the window changes its frame
accordingly.  In other words, if the window autosaves its frame, the
next time the window is created from a GSMarkup file, the autosaved frame
will override any other frame.  Last, the window is ordered front (by
using \texttt{orderFront:}), unless the attribute visible has been set
to no.

\subsubsection{Attributes}
\begin{itemize}
\item {\bf autosaveName}: A string (not localized).  If set, the window 
frameAutosaveName is set to it, and -- after the window has been
created and set to its default size and location --, the frame saved
under this name is automatically retrieved (if found) in the user
defaults, and the window frame is changed to it.  This means that the
first time the window is displayed, it will use the frame/size which
is determined by its content view, or which is hardcoded in the gsmarkup
file (the latter overriding the previous); but the next time, it will
resize and relocate itself to the same frame the user gave it last
time it was used.
\item {\bf backgroundColor}: A color.  If set, the window natural background
color is overridden by this color.
\item {\bf borderless}: A boolean.  If set to yes, the window is created without
any decorations.
\item {\bf center}: A boolean.  If set to yes, the window is automatically 
centered when it is created.
\item {\bf closable}: A boolean.  If set to no, the window is created non closable
(it is closable otherwise, unless borderless is set to yes).
\item {\bf contentHeight}: A positive float.  If set, the heigth of the window
content size is set to this.  This is done after the window has been
sized to fit its content view (if any), but before the window is sized
to the autosaved frame (if any).
\item {\bf contentWidth}: A positive float.  If set, the width of the window
content size is set to this.  This is done after the window has been sized
to fit its content view (if any), but before the window is sized to the
autosaved frame (if any).
\item {\bf heigth}: A positive float.  If set, the heigth
of the window frame is set to this value.  See comments for {\bf x}
about interaction with autosaved frame.
\item {\bf maxHeigth}: A positive float.  If set, the heigth
of the window maximum size is set to this value.
\item {\bf maxWidth}: A positive float.  If set, the width  
of the window maximum size is set to this value.
\item {\bf minHeigth}: A positive float.  If set, the heigth
of the window minimum size is set to this value.
\item {\bf minWidth}: A positive float.  If set, the width  
of the window minimum size is set to this value.
\item {\bf miniaturizable}: A boolean.  If set to no, the window is created 
non miniaturizable (it is miniaturizable otherwise, unless borderless is
set to yes).
\item {\bf releasedWhenClosed}: A boolean.  If set to yes, the window is set to be 
released when closed, if set to no, the window is set to not be released
when closed.  Please note that the default behaviour differs between windows
and panels: windows by default are released when closed, while panels are not.
\item {\bf resizable}: A boolean.  If set to no, the window is created 
non resizable (it is resizable otherwise, unless borderless is set to yes).
\item {\bf title}: A localizable string.  If set, the window title is set to it.
\item {\bf titled}: A boolean.  If set to no, the window is created without
a titlebar (it has a titlebar otherwise, unless borderless is set to yes).
\item {\bf visible}: A boolean.  If set to no, the window is not made visible
after it has been created.  Normally, all windows are automatically
ordered front (by calling \texttt{orderFront:}) after they are
created.  By setting visible to no, you can prevent this to happen.
\item {\bf width}: A positive float.  If set, the width  
of the window frame is set to this value.  See comments for {\bf x} about
interaction with autosaved frame. 
\item {\bf x}: A float.  If set, the x coordinate of the origin 
of the window frame is set to this value.  If the window autosaves its
frame, the autosave frame is set last, overriding any specification
found in the gsmarkup file ... which means setting this attribute affects
the window only the very first time it is displayed; later on, the
autosaved origin is used instead.
\item {\bf y}: A float.  If set, the y coordinate of the origin of 
the window frame is set to this value.  See comments for {\bf x} about
interaction with autosaved frame.
\end{itemize}

\subsubsection{Possible object attributes}
\begin{itemize}
\item {\bf delegate}: The window delegate.
\item {\bf initialFirstResponder}: The window initialFirstResponder.
\end{itemize}

\subsubsection{Content}
A \texttt{<window>} tag can contain a single tag, representing a view.
The platform object created by this tag is set to be the window
content view, and the window itself is sized (at first) so that the
content view exactly fits.  The window is later on resized if a fixed
content size or frame is specified in the attributes; finally, if the
window is autosaving its frame, the saved frame is restored last, so
that it overrides all other specifications.

\subsubsection{Examples}
Really depends on the type of window you are creating.  Typically you
want to set a title, change the window decorations (closable,
resizable, miniaturizable), set an autosaveName; from time to time,
you might also need to hardcode the contentSize of the window, or the
window frame.  If you are creating a panel, please use the
\texttt{<panel>} tag, which accepts all the attributes of a \texttt{<window>}
tag, but creates a \texttt{NSPanel} rather than a \texttt{NSWindow}
(and has a few panelish attributes more).  An example window with
nothing inside:
\begin{verbatim}
  <window title="My First Window"/>
\end{verbatim}
A window containing a button:
\begin{verbatim}
  <window title="Test" center="yes" resizable="no" autosaveName="terminate">
    <button title="Click to quit the application" action="terminate:">
  </window>
\end{verbatim}
please note that autosaving the frame will save the window location,
so it is a good idea to do it even if the window is not resizable.  A
window with a fixed frame:
\begin{verbatim}
<window title="frame of 500x100" autosaveName="fixedFrameExample"
        width="500" heigth="100"/>
\end{verbatim}

%%
%% Connector tag reference
%%

\section{Connector tag reference}

\subsection{The control tag}

\subsubsection{Description}
A \texttt{<control>} tag represents a control connector, and is
converted into an instance of \texttt{GSMarkupControlConnector}.  When the
connection is established, the following code is executed
\begin{verbatim}
  [source setAction: action];
  [source setTarget: target];
\end{verbatim}
where \texttt{source} is the source object, \texttt{target} is the
target object, and \texttt{action} is obtained by converting the
action attribute into a selector.

Because objects which allow you to set an action normally have an
action attribute, this connector is normally never used, since you
would rather set the action of the source by using the action
attribute of the source object, and then set the target of the source
to the target object by using a standard outlet connector with key
"target", or setting the object attribute target of the source to the
target object.  Please have a look at the examples which should make
this clear.

\subsubsection{Attributes}
\begin{itemize}
\item {\bf action}: The action of the connector.  This attribute is a string 
(non localizable), which is converted into a selector when the
connection is established.  If missing (or if the specified selector
can't be found), a \texttt{NULL} action is used.
\item {\bf label}: If there is no \texttt{action} but a \texttt{label} attribute
is present, it is used instead of \texttt{action}.
\item {\bf source}: A reference to another object -- a string beginning with
a hash (\texttt{\#}).  This object is the source of the control
connector.  If the string contains a dot, the object is looked up in
the name table, and then key-value coding is applied (see the
description of advanced outlets).  It is required.
\item {\bf target}: A reference to another object -- a string beginning with
a hash (\texttt{\#}).  This object is the target of the control
connector.  If the string contains a dot, the object is looked up in
the name table, and then key-value coding is applied (see the
description of advanced outlets).  If omitted, \texttt{nil} will be
used as target when establishing the connection.
\end{itemize}

\subsubsection{Examples}
Control connectors are used very rarely.  Here is how a control
connector would look:
\begin{verbatim}
<connectors>
  <control source="#myButton" target="#myController" action="buttonPressed:"/>
</connectors>
\end{verbatim}
this is rarely used, because usually you can more simply do
\begin{verbatim}
<objects>
  ...
    <button action="buttonPressed:" target="#myController" .../>
  ...
</objects>
\end{verbatim}
this is equivalent, but it is preferred because it is much more
natural: you set the action and the target of the button in the same
place where you create the button, rather than in the separate
connectors section.  The system automatically writes all action
connectors in this way when it writes gsmarkup files -- if it can: if the
source is not created in the gsmarkup file, this can't be done, and you
would need to use a control connector explicitly.

\subsection{The outlet tag}

\subsubsection{Description}
An \texttt{<outlet>} tag represents an outlet connector, and is converted
into an instance of \texttt{GSMarkupOutletConnector}.  When the connection is
established, the following code is executed
\begin{verbatim}
  [source takeValue: target  forKey: key];
\end{verbatim}
where \texttt{source} is the source object, \texttt{target} is the
target object, and \texttt{label} is the key.  Often an outlet can be
embedded directly into the objects section, which results in simpler,
better code.

\subsubsection{Attributes}
\begin{itemize}
\item {\bf key}: The key of the connector.  This attribute is a string 
(non localized), and is required (unless a label is provided).  It is
the key whose value is set to target, using key-value coding on
source.
\item {\bf label}: If there is no \texttt{key} but a \texttt{label} attribute
is present, it is used instead of \texttt{key}.
\item {\bf source}: A reference to another object -- a string beginning with
a hash (\texttt{\#}).  This object is the source of the outlet
connector.  If the string contains a dot, the object is looked up in
the name table, and then key-value coding is applied (see the
description of advanced outlets).  It is required.
\item {\bf target}: A reference to another object -- a string beginning with
a hash (\texttt{\#}).  This object is the target of the outlet
connector.  If the string contains a dot, the object is looked up in
the name table, and then key-value coding is applied (see the
description of advanced outlets).  If omitted, \texttt{nil} will be
used as target when establishing the connection.
\end{itemize}

\subsubsection{Examples}
Outlet connectors are not used often, but they are needed sometimes.
Here is how an outlet connector would look:
\begin{verbatim}
<connectors>
  <outlet source="#myController" target="#myButton" key="button"/>
</connectors>
\end{verbatim}
this (in which the source is an object already existing in the
application) is the only form of outlet connector which is normally
used, because if the object is created in the gsmarkup file itself, the outlet
can be embedded in the object creation by using the syntax
\begin{verbatim}
<objects>
  ...
    <textView delegate="#myController" .../>
  ...
</objects>
\end{verbatim}
which is perfectly equivalent to:
\begin{verbatim}
<objects>
  ...
    <textView id="myTextView" .../>
  ...
</objects>

<connectors>
  <outlet source="#myTextView" target="#myController" key="delegate"/>
</connectors>
\end{verbatim}
this second explicit form is much more long, artificial and
cumbersome.  The system automatically writes all outlet connectors
inside the objects section when it writes gsmarkup files -- if it can.  In
some cases (when the source is created outside the gsmarkup file) this
can't be done; these are the cases in which the connector is created
manually inside the connectors section.

\section{Example}

FIXME: this example should be replaced with a better one.

FIXME: add examples of advanced outlets.

\begin{verbatim}

<?xml version="1.0"?>
<!DOCTYPE gsmarkup>
<gsmarkup>

<objects>
 <!-- the main window -->
 <window title="Calculator" resizable="no" closable="no">

   <vbox>

     <textfield editable="no" id="TextField"/>

     <hbox>
      <button title="1" target="#Controller" action="digit:"/>
      <button title="2" target="#Controller" action="digit:"/>
      <button title="3" target="#Controller" action="digit:"/>
     </hbox>

     <hbox>
      <button title="4" target="#Controller" action="digit:"/>
      <button title="5" target="#Controller" action="digit:"/>
      <button title="6" target="#Controller" action="digit:"/>
     </hbox>

     <hbox>
      <button title="7" target="#Controller" action="digit:"/>
      <button title="8" target="#Controller" action="digit:"/>
      <button title="9" target="#Controller" action="digit:"/>
     </hbox>

     <hbox>
      <button title="+" target="#Controller" action="add:"/>
      <button title="0" target="#Controller" action="digit:"/>
      <button title="=" target="#Controller" action="total:"/>
     </hbox>

   </vbox>

 </window>

 <!-- an object of a custom class, which is instantiated when the nib
      is loaded.  Key-value coding is used to set textField to #TextField.
      In other words, the textField attribute is equivalent to having a
      NSNibOutletConnector with source #Controller, target #TextField, 
      and key textField -->
 <instance class="CalculatorController" id="Controller" 
           textField="#TextField"/>

</objects>


<!-- in this example, connectors are not actually needed (or better
     all connectors are embedded in the objects section) - but we add
     the connectors section as an example.  -->
<connectors>

 <!-- an example of a standalone connector, needed to connect the Controller
      to the controller ivar (or setController: method) of the NSOwner (the
      object which loaded the nib).  This connector must be standalone
      because #NSOwner is not inside the nib.  (you can reference objects
      totally outside the nib in a connector, which are provided by the
      nameTable dictionary when the nib is loaded) -->
 <outlet source="#NSOwner" target="#Controller" key="controller"/>

</connectors>

</gsmarkup>
\end{verbatim}
