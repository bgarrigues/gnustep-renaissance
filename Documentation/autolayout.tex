\chapter{Renaissance AutoLayout}

\section{What is the Renaissance AutoLayout package}
The Renaissance AutoLayout package provides advanced and friendly support
for automatically sizing and positioning objects (views) inside
windows at run time.  We generally call this work ``autolayout''.
This includes the operations of sizing and positioning.  In some cases
(typically when the user resizes an existing window, or when a
significant change is done to the objects displayed in a window), the
job of sizing and positioning is done for objects which had already
been sized and positioned before, and whose size and position is
somewhat updated because of new changes.  In this case, technically,
we rather talk of ``autoresizing'' and ``autorepositioning'', but we
still consider them special cases of ``autosizing'' and
``autopositioning'' (and so, of ``autolayout'').

\section{Why using the Renaissance AutoLayout support?}
The Renaissance AutoLayout package provides many advantages over
hard-coding the size and position of objects at compile time (as it is
done traditionally in NeXTstep NIB files):
\begin{enumerate}
\item the package can automatically adjust the layout for the different 
sizes of objects on different platforms or under different themes:
objects such as popup buttons have different sizes on GNUstep and on
Apple Mac OS X, so hardcoding the size and position of popup buttons
creates user interfaces which are not portable between the two
systems: a popup button which is perfectly sized on Apple Mac OS X might be
too tiny on GNUstep, and vice versa; as a matter of fact,
traditionally, you need to write separate user interfaces, with
hardcoded different sizes and positions, for the different systems.
\item the package can automatically adjust the layout for the different
sizes of objects when the strings displayed in the objects are
translated into another language.  Objects such as buttons, if sized
for the string in the original language, might be too small (or too
big) to display the string when translated into another language.
This is a basic problem of NIB files: if you use NIB files, you need
to create a separate NIB file for each language in which you want to
translate the user interface.
\item the package can automatically adjust the layout if objects are added
or removed from the window.
\item the package can automatically setup intelligent autorelayout 
behaviour of the objects, with minimum manual intervention from the
programmer.  Too much software has windows with inconsistent,
inaccurate or even meaningless behaviour under resizing.  The reason
is that checking and testing all the autoresizing flags and masks and
setup is a boring and stupid task, which most programmers would
happily avoid doing.  Renaissance AutoLayout will do it for you whenever
you are in a hurry; of course, you can still override manually any of
Renaissance's choices about autolayout whenever you need, and you will
probably need to give Renaissance a few simple align hints (center, left,
right) to make complex windows pretty.
\end{enumerate}

In comparison with other similar packages available for other
toolkits, Renaissance is at its best when dealing with ``standard''
windows.  Other toolkits usually provide you with tons of options for
the layout of each widget (or set of widgets), and you actually need
to understand and set a lot of those options (border, padding, expand,
fill, pack ...) just to get a standard window to layout correctly.
Programmers then have to continuously lookup in reference manuals the
meaning of most options, and they often enough forget to set some of
the options, ending up with software where the logic of the software
works, but widgets are laid out with slightly irregulars or
inconsistent layouts, often with irregular and inconsistent paddings or
alignments or autolayout behaviours.

At a more accurate examination, it is found out that most windows of
most software are actually similar and display very similar layout
patterns, and that there is a precise, mechanical logic which can be
used to get most autolayout options right.  Renaissance does its best to
use logic and intelligence to get the autolayout options right for
you, freeing you from the burden of setting manually all the
autolayout options, and allowing you to build a better interface for
your software, and much faster.


\section{The basics of autolayout}
Autolayout can be a very complex task.  There are only a few basic
requirements which all autolayout systems must implement:
\begin{enumerate}

\item different objects in the window should not overlap -- that is, 
the objects (for example buttons or textviews) should be displayed in
separate areas of the window, and not one over the other one, as that
would make part of one object disappear below the other one.

\item each object has -- usually -- a minimum size needed to display, 
and no object should ever be sized to be smaller than this minimum
size.  For example, a 'Cancel' button should never be made smaller
than the minimum size required to display the 'Cancel' string.  But
even this basic requirement is questioned in some autolayout systems
-- there are systems were buttons are allowed to become arbitrary
small (till you can no longer read what they are about, and possibly
not even push them!).  Still, you can get this effect even on systems
supporting a minimum size, by manually setting the minimum size of a
button to zero.  Renaissance AutoLayout does support (and encourage
the use of) minimum sizes.
\end{enumerate}

In theory, given those two basic requirements, not much more can be
said: there are infinite ways of organizing the (auto)layout of a
window.  After all a window is just a rectangular area, and there are
infinite ways you can layout your buttons and textfields inside this
rectangular area, infinite relationships you may want to establish
between the widths and heights and coordinates of those objects.

Renaissance provides you with support for some standard ways of organizing
your objects, which should be enough to build most standard windows
used in most software.  But there is always the possibility that you
could want/need to have some different relationships between the sizes
and positions of your objects.  If you are very much keen on
establishing ratios between heights and widths and positions of all
your objects, it's quite possible that you could end up with size and
position constrains which can't be implemented using the standard
Renaissance objects.  In that case, you will need to implement your own
autolayout objects, possibly subclassing or reusing in some other way
standard Renaissance AutoLayout objects to cut down your work.

\section{Introduction to basic Renaissance autolayout objects}
As in most other autolayout toolkits, most of the basic autolayout
objects in Renaissance are logical views which group different objects and
display them in rows, or columns, or tables.  These autolayout views
as often called ``containers'' in the literature.  They are
invisibile, logical objects, which contain other objects, and organize
the contained objects in rows, or columns, or tables, depending on the
type of container.

In the next few sections we will provide a short description of the
basic Renaissance autolayout objects.

\subsection{Boxes}
By far the most common containers are boxes.  Boxes are used to layout
objects in columns, or rows.  A horizontal box can layout objects in a
row; a vertical box can layout objects in a column.  A box object is
invisible; it is only a logical container which lays out the objects
it contains, and can be placed in a window wherever you can place a
normal object.  You add objects in a box, and the box automatically
gets bigger to accomodate the objects, and arranges the objects inside
itself to be on a row/column.

Naturally, boxes are themselves objects, so you can put boxes inside
other boxes.  A window usually contains a complete hierarchy of boxes,
organizing the objects into lines and columns, one into the other one.

Depending on the logic which the box uses to lay out the objects, it
can be of a different type.  In Renaissance, there are two basic types of
boxes: a ``standard'' box, and a ``proportional'' box.  The standard
box is the one used most often; the proportional box is normally used
only for special effects, when the views contained in the box must be
kept with sizes in specified proportions; a typical example is that of
the ``Cancel'' and ``OK'' buttons in panels, which should be of the
same size (the minimum size of the bigger of the two).  This can be
easily done by using a proportional box rather than a standard box to
layout the buttons (as will be explained in full details later).

In summary, in Renaissance there are four basic types of boxes:
horizontal standard boxes, horizontal proportional boxes, vertical
standard boxes, and vertical proportional boxes.

\subsection{Grids}
A grid is a container, which can layout its objects as if they were in
an invisible table.  Grids are more complex and rarer than boxes;
still, there are cases where the layout can be done using a grid, but
can't be done with boxes, no matter how many.

A grid organizes the views it contains into rows and columns; a box
itself can be considered a grid with a single row (horizontal box) or
a single column (vertical box).

TODO: talk more about grids

\subsection{Spaces}
Spaces are not containers -- they are invisible views, with a minimum
initial size of zero.  When more space is available, the spaces
expand, thus consuming space and creating areas of empty space in the
window, normally for alignment purposes.  They can be needed in some
cases, to control where the alignment or paddings or spacing of other
objects.  You can put a space wherever you can put any other view.

\section{The minimum size of a view}
Views are rectangular areas in a window; in GNUstep, they are
instances of NSView.  Normally, a view is used to display some
information.  Each view has a minimum size which is required in order
to display properly.  The minimum size could change if the attributes
of the view change; but usually it remains the same.

The minimum size is of critical important for an autolayout system,
because one of the main issues of the autolayout system is to make
sure that all views are given enough space to display themselves
properly; that is, that each view has never a size which is smaller
than its minimum size.

\section{Renaissance AutoLayout support for minimum sizes}
Renaissance AutoLayout adds two methods to NSView in order to support
minimum size:

\begin{itemize}
\item \texttt{-(void)sizeToFitContent}, which should be 
implemented in all views embedded in an autolayout window, and should
resize the view to the minimum size required to properly display its
current contents, with the current attributes.
\item \texttt{-(NSSize)minimumSizeForContent}, which should return the 
minimum size needed to display the view's current contents, with the
current attributes.
\end{itemize}

\texttt{-minimumSizeForContent} is somewhat more primitive than 
\texttt{-sizeToFitContent}, and the natural implementation of 
\texttt{-sizeToFitContent} would simply be
\begin{verbatim}
- (void) sizeToFitContent
{
  [self setFrameSize: [self minimumSizeForContent]];
}
\end{verbatim}

Unfortunately, the OpenStep API works the other way round: it does not
provide anything equivalent to \texttt{-minimumSizeForContent}, while
for most views and controls in the AppKit, it provides a
\texttt{-sizeToFit} which is often roughly equivalent to 
\texttt{-sizeToFitContent}, or enough to implement it modulo adjustments.

So the standard implementation of \texttt{-sizeToFitContent} for most
of the AppKit views and controls is done by calling
\texttt{-sizeToFit} (but this needs to be reviewed for every control),
while the default NSView's implementation of
\texttt{-minimumSizeForContent} works by saving the view's frame, then
resizing the view to its minimum frame by calling
\texttt{-sizeToFitContent}, grasping the resulting frame, then setting
back the view's frame to its original size.

Because of this very clumsy and inefficient implementation, it is
recommended that you try to avoid using
\texttt{-minimumSizeForContent} if you can.

\section{Expanding views over their minimum size}
Some views display information, and once they are of the minimum size
required to display the information, making them bigger does not show
any more information.  A typical example is a fixed text label, or a
button with written 'Cancel', or a button with an image of fixed size.
If all the window were made of such views, once the window is laid
out, there would be no reason to make the window resizable.  All the
information is displayed comfortably on the window in the minimum
size, and more space on the window would help in no way, except mess
up the layout.

Other views display information, but if they are made bigger, they
benefit from it and can use the new space to display more information.
A typical example is an editable text field, where the user can insert
a string of arbitrary length.  When the window is automatically
created, the text field is of some arbitrary, predermined, length.  If
the user needs to insert more text than the text field can accomodate
at its default size, the user might want to enlarge the text field.
This is normally achieved by making the window resizable, and setting
up the autolayout in such a way that if the user makes the window
bigger, the text field is automatically made bigger.

Finally, often enough sizing a view to its minimum size is enough to
display the information in the view, but it looks pretty ugly in the
context of the other views.  This is true even for views which are
naturally and inherently unresizable; for example, in panels it's
common to display the 'OK' and the 'Cancel' button at the bottom, with
the same size.  Even if the minimum size of the 'OK' and of the
'Cancel' button are different and they are never supposed to be
resized when the window is later resized, it's more aestethically
pleasant to make them of the same size.

In other cases, views which look nice at their minimum size might
still require alignment to fit nicely in the rest of the window.  If
more space is available, while they are not resized themselves, they
might need the space to be spread intelligently around in order to be
moved into the right position.

Finally, some views do look nice at their minimum size, but it might
be graphically nice to expand them when the window is expanded over
the minimum size, to keep them in a pleasant relationship with the
other views.

\section{How expanding flags work}
The Renaissance AutoLayout framework makes extensive use of the
concept that some views (the ones which are not necessarily able to
display all information when they are at their minimum size) benefit
from being expanded, while other views (the ones which always display
all information when they are at their minimum size) do not benefit
from being expanded - actually they are better not expanded since we
don't know where to put the additional space :-) and we just make the
layout uglier.

Please note that views can benefit from being expandable in one
direction, but not in the other one.  For example, an editable
textfield normally displays a single line of text, so it benefits from
being expanded in the horizontal direction, but not in the vertical
one.

\section{AutoLayout flags}\label{autolayout-flags}
Each view has two expand/align flags, one for the horizontal, and the
other one for the vertical direction.  Such flags describe how the
view should be treated with respect to additional space which could be
made available.  'Additional space' means 'space in addition to the
minimum required to display the view'.  The flags for a view are
generally managed (and stored) in the container which contains the
view.  Different containers will react to flags in different ways,
particularly to the align values of the flags.

Each of the horizontal and vertical flag of a view can take a value
from the following enumeration:
\begin{verbatim}
typedef enum 
{
  GSAutoLayoutExpand = 0,
  GSAutoLayoutWeakExpand = 1,
  GSAutoLayoutAlignBottom = 2,
  GSAutoLayoutAlignCenter = 3,
  GSAutoLayoutAlignTop = 4
} GSAutoLayoutAlignment;
\end{verbatim}

These values are now described fully.
\begin{itemize}

\item \texttt{GSAutoLayoutExpand}: The view benefits from being expanded
in that direction, typically it can provide a bigger editable area
and/or display more useful information.  Editable textfields (in the
horizontal direction) and scrollviews (in both directions) are typical
examples of controls which have this flag.  Containers have their flag
set to \texttt{GSAutoLayoutExpand} in a direction if and only if they
contain a view with the flag set to \texttt{GSAutoLayoutExpand} in
that direction.  Typically, a window is then made resizable in a
direction if and only if its main container has this value of the
align flag in that direction.  This automatically makes windows which
contains at least an element which benefits from being expanded in a
direction expandable in that direction.

\item \texttt{GSAutoLayoutWeakExpand}:  The view does not benefit from being
expanded in that direction, but whenever more space is available, it
looks prettier if the view is expanded.  This value of the flag is
typically used by `spaces'.  A space is an invisible view whose task
is just to expand and consume some space when the space becomes
available (usually in order to keep other controls aligned and
prettily positioned).  A container has a flag set to
\texttt{GSAutoLayoutWeakExpand} in a direction if and only if it does 
not contain any view with the flag set to \texttt{GSAutoLayoutExpand}
(else, its flag would be set to this), and it contains a view with the
flag set to \texttt{GSAutoLayoutWeakExpand}.

\item \texttt{GSAutoLayoutAlignCenter}: The view does not benefit from being
expanded in that direction, and if more space is available, the view
would prefer the space to be distributed on the two sides of the view,
keeping the view centered in that direction.  The actual
interpretation of this flag depends on the container, which might
ignore it (as the case of the proportioned boxes, which always expands
all the contained views in their direction).  This value is the
default for all views which don't have the flag set to expand.

\item \texttt{GSAutoLayoutAlignBottom}, \texttt{GSAutoLayoutAlignTop}:
The view does not benefit from being expanded in that direction, and
if more space is available, the view would prefer the space to be
distributed on one side of the view, so that the view is aligned
towards the bottom (minimum) or top (maximum) of the coordinate system
in that direction.

\end{itemize}

\section{How AutoLayout flags are determined}
When a container manages a view, it has to determine the align flags
of the view.  When the view is first added, the container calls the
method \texttt{-autoLayoutDefaultHorizontalAlignment} of the view, to
get its horizontal align flag, and the method
\texttt{-autoLayoutDefaultVerticalAlignment} to get its vertical align
flag.  If the values returned by those methods are not appropriate, it
is possible to change the flags by calling methods of the container
manually for each view (for example, for boxes, by calling the method
\texttt{-setHorizontalAlignment:forView:}).  Any flag manually set will
override the default flags.

Renaissance AutoLayout adds a category to NSView, implementing the
following methods:

\begin{itemize}
\item \texttt{- (GSAutoLayoutAlignment)
  autoLayoutDefaultHorizontalAlignment}, which should return the
  default alignment in the horizontal direction for that view (given
  the view's class and attributes).
\item \texttt{- (GSAutoLayoutAlignment)
  autoLayoutDefaultVerticalAlignment}, which should return the default
  alignment in the vertical direction for that view (given the view's
  class and attributes).
\end{itemize}

The default implementation of those methods in \texttt{NSView} returns
\texttt{GSAutoLayoutAlignCenter} for both of them; Renaissance AutoLayout 
also provides implementations of those methods (as categories) for
most standard AppKit classes; generally returning
\texttt{GSAutoLayoutExpand} if the control benefits from being expanded 
in a direction, and \texttt{GSAutoLayoutAlignCenter} otherwise.

You should implement those methods in your subclasses to make sure
they interact properly with Renaissance AutoLayout.

\section{Paddings}
Some containers allow your views to have paddings set.  When the view
is placed inside the container, additional space will be reserved
around the view for the paddings; that space will be left intentionally
blank.

The container will initially call the methods
\begin{verbatim}
-autoLayoutDefaultLeftPadding
-autoLayoutDefaultRightPadding
-autoLayoutDefaultBottomPadding
-autoLayoutDefaultTopPadding
\end{verbatim}
of the view to get its default padding on the various sides.

You can later change the paddings manually by calling the appropriate
method of the container, such as \texttt{-setLeftPadding:forView:} for
boxes.

Renaissance AutoLayout adds a category to NSView, implementing the
following methods:

\begin{itemize}
\item \texttt{- (float) autoLayoutDefaultLeftPadding}, which should
return the default padding on the left side for that view.
\item \texttt{- (float) autoLayoutDefaultRightPadding}, which should
return the default padding on the right side for that view.
\item \texttt{- (float) autoLayoutDefaultBottomPadding}, which should
return the default padding on the bottom side for that view.
\item \texttt{- (float) autoLayoutDefaultTopPadding}, which should
return the default padding on the top side for that view.
\end{itemize}

The default implementation of those methods in \texttt{NSView} returns
\texttt{4} for all of them.  Containers and spaces have
implementations of those methods returning \texttt{0} for all of them.

Unless you are looking for special effects (or for trouble), you
generally don't need to modify paddings.
